\capitulo{1}{Introducción}

Un pilar básico en las organizaciones públicas y privadas que se dirigen a la sociedad, ya sea ofreciendo un producto o un servicio, es la opinión de las personas. 

En el pasado la forma de conocer estas opiniones era mediante encuestas que después se trataban para hacer estadísticas y evaluar la calidad del producto.

Estas encuestas tiene dos grandes limitaciones:
\begin{itemize}
\tightlist
    \item La muestra es limitada y a partir de esta debemos generalizar.
    \item La opinión de esta población viene dada por unas preguntas que realiza la organización previamente, por tanto son demasiado específicas y puede que haya aspectos que no se tengan en cuenta.
\end{itemize}

Hoy en día las redes sociales constituyen un gran avance en este ámbito. La gente puede mostrar su opinión sobre cualquier servicio o producto sin necesidad de encuestas y además pueden tener un alcance global.
Esto representa un gran cambio tanto para las personas como para las organizaciones. La gente puede basarse en la opinión de personas muy dispares para tomar decisiones sobre un producto. Las organizaciones pueden aprovecharse de la diversidad de la población y de la sinceridad de la opinión ya que no está acotada por preguntas exactas.

En este punto es donde el análisis de sentimiento toma protagonismo. Este concepto no es más que el estudio del lenguaje natural, análisis de texto y lingüística computacional, para tratar de establecer métricas capaces de capturar el sentimiento general de un texto. 

Es un campo muy grande que abarca diferentes técnicas. Como ocurre con otros términos, es más conocido el nombre en inglés, Sentiment Analysis, por lo que a partir de ahora nos referiremos al tema con el término anglosajón.

Estas métricas han hecho que las organizaciones complementen sus encuestas con los resultados obtenidos en las redes sociales y así poder tener un mayor feedback sobre sus servicios. 

En este trabajo se propone la utilización de Sentiment Analysis para realizar estudios en Twitter e Instagram sobre temas de actualidad, productos o servicios. 
Para ello se recopilará información de estas redes sociales y se mostrará en gráficos y tablas para hacer más fácil la interpretación de los datos.



\section{Estructura de la memoria}
\begin{itemize}
\tightlist
    \item
        \textbf{Introducción: }Descripción de la situación y el tema sobre el que el proyecto va a versar. Estructura de la memoria y de los anexos.
    \item 
        \textbf{Objetivos del proyecto: }Explicación de los temas a tratar en el trabajo.
    \item 
        \textbf{Conceptos teóricos: }Exposición de conceptos que facilitan la comprensión del proyecto.
    \item 
        \textbf{Técnicas y herramientas: }Listado de metodologías y herramientas que han sido utilizadas para llevar a cabo el proyecto. 
    \item 
        \textbf{Aspectos relevantes: }Muestra aspectos a destacar durante la realización del proyecto
    \item 
        \textbf{Trabajos relacionados: }Estado del arte en el ámbito del sentiment analysis y trabajos similares.
    \item 
        \textbf{Conclusiones y líneas de trabajo futuras: }Conclusiones obtenidas al final del proyecto y posibles ideas futuras.
\end{itemize}

\newpage
\section{Estructura de los anexos}
\begin{itemize}
\tightlist
    \item 
        \textbf{Plan de proyecto software: }Planificación temporal y viabilidad económica y legal.
    \item 
        \textbf{Especificación de requisitos: }Objetivos y y requisitos establecidos al comienzo del proyecto.
    \item 
        \textbf{Especificación de diseño: }Recoge los diseños de datos, procedimental, arquitectónico y de interfaces.
    \item 
        \textbf{Manual del programador: }Explica los conceptos más técnicos del proyecto como su instalación, la organización de carpetas y la ejecución.
    \item 
        \textbf{Manual de usuario: }Es la guía de cómo utilizar la aplicación paso a paso.
\end{itemize}