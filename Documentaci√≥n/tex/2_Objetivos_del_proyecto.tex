\capitulo{2}{Objetivos del proyecto}

En este apartado vamos a detallar los objetivos que queremos abordar en el proyecto, que serán denominados generales, y los objetivos necesarios para llevar a cabo los anteriores, que denominaremos como técnicos.

\section{Objetivos generales}
\begin{itemize}
\tightlist
    \item Extraer datos de las redes sociales Twitter e Instagram.
    \item Estudiar mediante Sentiment Analysis la información recogida de estas APIs.
    \item Realizar estadísticas sobre los datos recopilados.
    \item Crear una aplicación web para mostrar la información.
    \item Mostrar las estadísticas mediante gráficos y tablas.
    \item Analizar mediante series temporales tendencias futuras en la evolución de las métricas.
    \item Crear una wiki para el manual de usuario.
\end{itemize}

\clearpage

\section{Objetivos técnicos}
\begin{itemize}
\tightlist
    \item Extraer los datos de las APIs de Twitter e Instagram.
    \item Utilizar librerías de Sentiment Analysis para analizar los datos obtenidos.
    \item Realizar operaciones estadísticas sobre los resultados de Sentiment Analysis.
    \item Almacenar en la base de datos los datos de las APIs junto a sus resultados y cálculos de las operaciones estadísticas.
    \item Desarrollar una interfaz que sea adaptativa y fácil de utilizar para el usuario.
    \item La aplicación deberá mostrar los resultados en gráficos y tablas.
    \item Utilizar Git como sistema de control de versiones junto con la plataforma GitHub.
    \item Usar la plataforma gráfica GitKraken para Git.
    \item Hacer uso de ZenHub para la gestión del proyecto.
    \item Utilizar un sistema para las referencias bibliográficas como Zotero.
    \item Utilizar la librería statsmodels de Python para calcular series temporales y realizar test para ver si es estacionaria.
\end{itemize}