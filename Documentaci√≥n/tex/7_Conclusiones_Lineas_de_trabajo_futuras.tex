\capitulo{7}{Conclusiones y Líneas de trabajo futuras}

\section{Conclusiones}
Una vez finalizado el proyecto  y remontándome al comienzo del mismo puedo afirmar que he adquirido muchos conocimientos.

El resultado de la aplicación cumple con todos los objetivos marcados inicialmente y además se han añadido implementaciones que no se pensaron en un primer momento pero que dan al proyecto mayor funcionalidad.

Por ejemplo, en un primer momento se pensó que la aplicación simplemente iba a obtener los resultados que estuvieran en la base de datos ya almacenados. Pero a mitad del proyecto se pensó que podía ser muy interesante tener una opción de 'Actualizar la base de datos' y también la opción de buscar un identificador que no se encontrara almacenado.
De esta forma, nuestra aplicación accede a las APIs de Twitter e Instagram en tiempo real y recolecta datos nuevos para almacenarlos en la base de datos.

He aprendido mucho sobre las aplicaciones web, los protocolos utilizados y las peticiones a través de ellos. En cuanto a los lenguajes, he reforzado los conocimientos que tenía sobre Python y SQL, ya que se cursan a lo largo del grado, y he aprendido el lenguaje TypeScript. También he aprendido sobre HTML, sobre el cual tenía unos conocimientos muy vagos.

Personalmente, lo que más me ha costado y creo que he aprendido mucho, ha sido cómo conectar la interfaz de usuario de una aplicación con el controlador de la misma. Para mí utilizar servicios y realizar peticiones era algo totalmente nuevo, pero al final he conseguido aprender, y aunque me queda mucho por mejorar, he conseguido realizar las peticiones de forma exitosa.

En cuanto a las metodologías, las cuales se tratan en la asignatura de gestión de proyectos, se ha usado Scrum que actualmente está muy presente en la mayoría de los trabajos. Utilizarla ha sido clave para conseguir una buena organización y aprender a estimar cuanto tiempo gastamos en cada tarea.

Por todo ello, creo que el balance ha sido bueno aun con los problemas que se han tenido durante el desarrollo del proyecto.

\section{Líneas de trabajo futuras}

\begin{itemize}
    \item Crear varios roles de usuario.
    \item Implementar el acceso con cuentas de otras aplicaciones como Google, Facebook, etc.
    \item Implementar seguridad en la aplicación.
    \item Realizar sentiment analysis mediante técnicas de machine learning.
    \item Añadir más redes sociales de las que recabar información.
    \item Utilizar scraping para obtener información.
\end{itemize}